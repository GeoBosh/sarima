\documentclass[article,nojss]{jss}
\usepackage{amsmath, amsfonts}

%% need no \usepackage{Sweave}
%\VignetteIndexEntry{Autocorrelations and white noise tests}


\author{Georgi N. Boshnakov \\ University of Manchester }
\Plainauthor{Georgi N. Boshnakov}

\title{Autocorrelations and white noise tests}
\Plaintitle{Autocorrelations and white noise tests} %% without formatting
%\Shorttitle{} %% a short title (if necessary)

%% an abstract and keywords
\Abstract{
    The \proglang{R} package \proglang{sarima} provides
    functions, classes and methods for time series modelling with ARIMA
    and related models. The aim of the package is to provide consistent
    interface for the user. This document gives examples for white noise tests.
}
\Keywords{arima, sarima, time series, \proglang{S4}, \proglang{R}}
\Plainkeywords{arima, sarima, time series, S4, R} %% without formatting
%% at least one keyword must be supplied

%% publication information
%% NOTE: Typically, this can be left commented and will be filled out by the technical editor
%% \Volume{50}
%% \Issue{9}
%% \Month{June}
%% \Year{2012}
%% \Submitdate{2012-06-04}
%% \Acceptdate{2012-06-04}

%% The address of (at least) one author should be given
%% in the following format:
\Address{
  Georgi N. Boshnakov\\
  School of Mathematics\\
  The University of Manchester\\
  Oxford Road, Manchester M13 9PL, UK\\
  URL: \url{http://www.maths.manchester.ac.uk/~gb/}
}


\begin{document}
%% include your article here, just as usual
%% Note that you should use the \pkg{}, \proglang{} and \code{} commands.

% borrowed from zoo.Rnw in package zoo:


%\VignetteIndexEntry{Brief guide to package sarima}
%\VignetteDepends{}
%\VignetteKeywords{portmanteau tests, autocorrelations, ARIMA, time series, S4, R}
%\VignettePackage{sarima}



\section{Autocorrelations and related properties}
\label{sec:autoc-relat-prop}

The generic function \code{autocorrelations()} computes autocorrelations. What exactly is
computed depends on the first argument. There are analogous functions for other second order
characteristics, e.g. \code{partialAutocorrelations}, see the package documentation for the
available functions.

The examples below use models used in the examples in Francq \& Zakoian's book on GARCH
models. It can be consulted for concepts and technical details.  

\begin{Schunk}
\begin{Sinput}
R> n <- 100
R> ma2.model <- list(ma = c(0.56, -0.44))
R> xma2 <- arima.sim(ma2.model, n)
\end{Sinput}
\end{Schunk}

With time series or similar argument \code{autocorrelations} computes the sample
quantity. There are similar functions for other quantities, the example below computes also
partial autocorrelations:
\begin{Schunk}
\begin{Sinput}
R> xma2.acf <- autocorrelations(xma2, maxlag = 8)
R> class(xma2.acf)
\end{Sinput}
\begin{Soutput}
[1] "SampleAutocorrelations"
attr(,"package")
[1] "sarima"
\end{Soutput}
\begin{Sinput}
R> xma2.pacf <- partialAutocorrelations(xma2, maxlag = 8)
R> class(xma2.pacf)
\end{Sinput}
\begin{Soutput}
[1] "SamplePartialAutocorrelations"
attr(,"package")
[1] "sarima"
\end{Soutput}
\end{Schunk}

If the argument is a model, the suitable theoretical properties are computed.
In interactive use an ARMA model can be specified also as a list with components \code{ar},
\code{ma} and \code{sigma2}

\begin{Schunk}
\begin{Sinput}
R> xma2.tacf <- autocorrelations(ma2.model, maxlag = 8)
R> class(xma2.tacf)
\end{Sinput}
\begin{Soutput}
[1] "Autocorrelations"
attr(,"package")
[1] "sarima"
\end{Soutput}
\begin{Sinput}
R> xma2.tpacf <- partialAutocorrelations(ma2.model, maxlag = 8)
R> class(xma2.tpacf)
\end{Sinput}
\begin{Soutput}
[1] "PartialAutocorrelations"
attr(,"package")
[1] "sarima"
\end{Soutput}
\end{Schunk}

A plot of the autocorrelation object \code{plot} can be used to check if the time series is
white noise. Without further arguments, the confidence limits correspond to a null hypothesis
of iid:
\begin{Schunk}
\begin{Sinput}
R> plot(xma2.acf)